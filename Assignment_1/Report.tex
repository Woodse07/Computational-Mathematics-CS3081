\documentclass[12pt]{report}
\usepackage{amsmath}
\usepackage{graphicx}
\usepackage{hyperref}
\usepackage[utf8]{inputenc}
\usepackage{listings}

\title{CS3081 Computational Mathematics}
\author{Séamus Woods \\ 15317173}
\date{19/02/2019}

\begin{document}
\maketitle
\newpage

\section{Question 2.31}
Question: Write a user-defined MATLAB function that calculates the determinant of a square ( n x  n) matrix, where n can be 2, 3, or 4. For function name and arguments, use D =  Determinant (A). The input argu­ment A is the matrix whose determinant is calculated. The function Determinant should first check if the matrix is square. If it is not, the output D should be the message "The matrix must be square." Use Determinant to calculate the determinant of the following two matrices: \[
\stackrel{\mbox{(a)}}{%
\begin{bmatrix}
{1} & {5} & {4} \\
{2} & {3} & {6} \\
{1} & {1} & {1} 
\end{bmatrix}%
}
\quad
\stackrel{\mbox{(b)}}{%
\begin{bmatrix}
{1} & {2} & {3} & {4} \\
{5} & {6} & {7} & {8} \\
{9} & {10} & {11} & {12} \\
{13} & {14} & {15} & {16} 
\end{bmatrix}%
}
\]
\newline
Part (a):
\begin{itemize}
\item[(i)] 4
\item[(ii)] 13
\item[(iii)] 26
\item[(iv)] 18
\end{itemize}
\textbf{Your Answer:}
\newline
The answer I got for part (a) was (ii)..13.
\newline
\newline
Part (b):
\begin{itemize}
\item[(i)] 0
\item[(ii)] 12
\item[(iii)] 7
\item[(iv)] 4
\end{itemize}
\textbf{Your Answer:}
\newline
The answer I got for part (b) was (i)..0.
\newline
\newline
My MATLAB code for calculating these answers can be seen below.
\lstinputlisting[language=Octave]
{2.31.m}
\section{Question 3.2}
Question: Determine the root of $f(x) = x - 2e^{-x}$ by: 
\begin{itemize}
\item[(a)] Using the bisection method. Start with a= 0 and  b = 1, and carry out the first three iterations.
\item[(b)] Using the secant method. Start with the two points, x1 = 0 and x2 = 1, and carry out the first three iter­ations. 
\item[(c)] Using Newton's method. Start at x1 = 1 and carry out the first three iterations. 
\end{itemize}
Part (a):
\begin{itemize}
\item[(i)] 0.1241
\item[(ii)] 0.08125
\item[(iii)] 0.074995
\item[(iv)] 0.003462
\end{itemize}
\textbf{Your Answer:}
\newline
Bisection Method: is a bracketing method for finding a numerical solution of an equation of the form $f(x) = 0$ when it is known that withing a given interval $[a,b], f(x)$ is continuous and the equation has a solution.
\newline
The algorithm for the bisection method is as follows:
\begin{itemize}
\item[1.] Choose first interval by finding points $a$ and $b$ such that a solution exists between them ($a$ and $b$ should have different signs). For us, $a$ and $b$ have been given to us as 0 and 1 respectively.
\item[2.] Calculate the first estimate of the numerical solution $x_{NS1}$ by:
\begin{center}
$x_{NS1} = \frac{(a+b)}{2}$
\end{center}
\item[3.] Determine if the solution is between a and $x_{NS1}$ or b and $x_{NS1}$. This is done by checking the sign of the product $f(a) * f(x_{NS1})$. If the result of this is less than 0, the solution is between a and $x_{NS1}$, else if the solution is greater than 0, the solution is between $x_{NS1}$ and b.
\item[4.] Select the subinterval that contains the true solution and go back to step 2. Step 2 through 4 are repeated until error bound is attained.
\end{itemize}
Since we have step 1 already done for us we will begin with step 2.
\begin{itemize}
\item[Iteration 0:] $x_{NS1} = \frac{(0+1)}{2} = 0.5$. This is our first estimate of our numerical solution. $f(0) * f(0.5) = ((0) - 2e^{-(0)}) * ((0.5) - 2e^{-(0.5)}) = -2 * -0.7130 = 1.426$. Since this is greater than 0, we know our solution is in between $x_{NS1}$ and b.
\item[Iteration 1:] $x_{NS1} = \frac{(0.5+1)}{2} = 0.75$. This is our second estimate of our numerical solution. $f(0.5) * f(0.75) = ((0.5) - 2e^{-(0.5)}) * ((0.75) - 2e^{-(0.75)}) = -0.7130 * -0.1947 = 0.1388$. Since this is greater than 0, we know our solution is in between $x_{NS1}$ and b.
\item[Iteration 2:] $x_{NS1} = \frac{(0.75+1)}{2} = 0.875$. This is our third estimate of our numerical solution. $f(0.75) * f(0.875) = ((0.75) - 2e^{-(0.75)}) * ((0.875) - 2e^{-(0.875)}) = -0.1947 * 0.04127 = -0.0080$. Since this is less than 0, we know our solution is in between a and $x_{NS1}$.
\item[Iteration 3:] $x_{NS1} = \frac{(0.75+0.875)}{2} = 0.8125$. This is our final estimate of our numerical solution. $f(0.75) * f(0.8125) = ((0.75) - 2e^{-(0.75)}) * ((0.8125) - 2e^{-(0.8125)}) = -0.1947 * -0.07499 = -0.0146$. Since this is less than 0, we know our solution is in between $x_{NS1}$ and a.
\end{itemize}
The answer we end up with is 0.8125.. or (ii)
\newline
\newline
Part (b):
\begin{itemize}
\item[(i)] 0.72481
\item[(ii)] 0.86261
\item[(iii)] 0.62849
\item[(iv)] 0.17238
\end{itemize}
\textbf{Your Answer:}
\newline
Secant Method: is a scheme for finding a  numerical solution of an equation of the form $f(x) = 0$. The method uses two points in the neigh­borhood of the solution to  determine a new estimate for the solution. Two points are used to define a straight line, and the point where the line intersects the x-axis is the new estimate for the solution. 
\newline
The equation can be generalized to an iteration formula in which a new estimate of the solution $x_{i+1}$ is determined from the previous two solutions $x_i$ and $x_{i-1}$
\begin{center}
$x_{i+1} = x_i - \frac{f(x_i)(x_{i-1}-x_i)}{f(x_{i-1})-f(x_i)}$
\end{center}
\begin{itemize}
\item[Iteration 1:] Let $x_i = b..(1)$ and $x_{i-1} = a..(0)$. We first find our next estimate of the solution by subbing into our formula.. $x_{i+1} = 1 - \frac{f(1)(0-1)}{f(0)-f(1)}$, giving us $x_{i+1} = 0.88339$. $f(0.88339) = 0.05663$.
\item[Iteration 2:] We now repeat the process for our new estimate of the solution. $x_{i+1} = 0.88339 - \frac{f(0.88339)(1-0.88339)}{f(1)-f(0.88339)}$, giving us $x_{i+1} = 0.85154$. $f(0.85154) = -0.00197$.
\item[Iteration 3:] And again.. $x_{i+1} = 0.85154 - \frac{f(0.85154)(0.88339-0.85154)}{f(0.88339)-f(0.85154)}$, giving us $x_{i+1} = 0.85261$. $f(0.85261) = 0.00000833298$.
\end{itemize}
So our answer is 0.85261 or (ii).. probably some inaccuracies due to rounding.
\newline
\newline
\newline
Part (c):
\begin{itemize}
\item[(i)] 0.65782
\item[(ii)] 0.59371
\item[(iii)] 0.45802
\item[(iv)] 0.85261
\end{itemize}
\textbf{Your Answer:}
\newline
Newton's method is a scheme for finding a numerical solution of an equation of the form $f(x) = 0 $ where $f(x)$ if continuous and differentiable and the equation is known to have a solution near a given point. The equation can be generalized for determining the "next" solution $x_{i+1}$ from the present solution $x_i$:
\begin{center}
$x_{i+1} = x_i - \frac{f(x_i)}{f^\prime(x_i)}$
\end{center}
\begin{itemize}
\item[Iteration 1:] First easiest to find out what $f^\prime(x)$ is.. $f^\prime(x) = 2e^{-x} + 1$. We know that $x_i$ = 1, so we just need to plug it into our formula to get the next solution. $x_{i+1} = 1 - \frac{f(1)}{f^\prime(1)}$ = 0.848.
\item[Iteration 2:] $x_{i+1} = 0.848 - \frac{f(0.848)}{f^\prime(0.848)}$ = 0.8433.
\item[Iteration 3:] $x_{i+1} = 0.8433 - \frac{f(0.833)}{f^\prime(0.833)}$ = 0.852. $f(0.852) = -0.0011$.
\end{itemize}
So our answer is 0.852 or (iv).
\newline
\newline


\section{Question 4.24}
Question: Write a user-defined MATLAB function that determines the inverse of a matrix using the Gauss-Jor­dan method. For the function name and arguments use Ainv =Inverse (A), where A is the matrix to be inverted, and Ainv is the inverse of the matrix. Use the Inverse function to calculate the inverse of: \[
\stackrel{\mbox{The Matrix A}}{%
\begin{bmatrix}
{-1} & {2} & {1} \\
{2} & {2} & {-4} \\
{0.2} & {1} & {0.5} 
\end{bmatrix}%
}
\quad
\stackrel{\mbox{The Matrix B}}{%
\begin{bmatrix}
{-1} & {-2} & {1} & {2} \\
{1} & {1} & {-4} & {-2} \\
{1} & {-2} & {-4} & {-2} \\
{2} & {-4} & {1} & {-2} \\
\end{bmatrix}%
}
\]
\begin{itemize}
\item[(i)] \[
\stackrel{\mbox{Inverse(a)}}{%
\begin{bmatrix}
{-0.7143} & {0.0} & {1.4286} \\
{0.2571} & {0.1000} & {0.2857} \\
{-0.2286} & {-0.2000} & {0.8571} 
\end{bmatrix}%
}
\quad
\stackrel{\mbox{Inverse(b)}}{%
\begin{bmatrix}
{1.6667} & {2.8889} & {-2.2222} & {1.0000} \\
{0.0} & {0.3333} & {-0.3333} & {0.0} \\
{-0.3333} & {-0.4444} & {0.1111} & {0.0} \\
{1.5000} & {2.0000} & {-1.5000} & {0.5000} 
\end{bmatrix}%
}
\]
\item[(ii)] \[
\stackrel{\mbox{Inverse(a)}}{%
\begin{bmatrix}
{0.7243} & {0.0} & {1.3286} \\
{1.2571} & {0.1000} & {0.2757} \\
{-0.2386} & {-0.2010} & {0.9571} 
\end{bmatrix}%
}
\quad
\stackrel{\mbox{Inverse(b)}}{%
\begin{bmatrix}
{1.6677} & {2.9889} & {3.2222} & {1.01700} \\
{0.3433} & {-0.3433} & {0.3333} & {0.00371} \\
{-0.3433} & {-0.2879} & {0.2111} & {0.0} \\
{1.2400} & {2.0120} & {-1.5783} & {0.5600} 
\end{bmatrix}%
}
\]
\item[(iii)] \[
\stackrel{\mbox{Inverse(a)}}{%
\begin{bmatrix}
{0.7143} & {0.003} & {2.3276} \\
{1.2671} & {0.1100} & {0.3759} \\
{-0.2486} & {-0.2110} & {0.9771} 
\end{bmatrix}%
}
\quad
\stackrel{\mbox{Inverse(b)}}{%
\begin{bmatrix}
{1.6877} & {3.9789} & {3.2002} & {2.01800} \\
{0.3533} & {-0.4433} & {0.3333} & {0.02371} \\
{-0.3443} & {-0.2999} & {0.3121} & {0.0382} \\
{1.2420} & {3.0130} & {-1.5733} & {0.5610} 
\end{bmatrix}%
}
\]
\item[(iv)] \[
\stackrel{\mbox{Inverse(a)}}{%
\begin{bmatrix}
{0.8343} & {1.01} & {1.3336} \\
{2.2572} & {0.1003} & {0.3857} \\
{-0.2486} & {-0.2110} & {0.9671} 
\end{bmatrix}%
}
\quad
\stackrel{\mbox{Inverse(b)}}{%
\begin{bmatrix}
{1.6777} & {4.9889} & {3.2232} & {1.11700} \\
{0.3443} & {-0.3443} & {0.3233} & {0.07371} \\
{-0.3443} & {-0.2979} & {0.3211} & {0.07800} \\
{1.2480} & {2.1220} & {-1.5883} & {0.5621} 
\end{bmatrix}%
}
\]
\end{itemize}
\textbf{Your Answer:}
\newline
The answer I got was (a) \[
\stackrel{\mbox{Inverse(a)}}{%
\begin{bmatrix}
{-0.7143} & {0.0} & {1.4286} \\
{0.2571} & {0.1000} & {0.2857} \\
{-0.2286} & {-0.2000} & {0.8571} 
\end{bmatrix}%
}
\quad
\stackrel{\mbox{Inverse(b)}}{%
\begin{bmatrix}
{1.6667} & {2.8889} & {-2.2222} & {1.0000} \\
{0.0} & {0.3333} & {-0.3333} & {0.0} \\
{-0.3333} & {-0.4444} & {0.1111} & {0.0} \\
{1.5000} & {2.0000} & {-1.5000} & {0.5000} 
\end{bmatrix}%
}
\]
\newline
\newline
My MATLAB code for calculating this can be seen below.
\lstinputlisting[language=Octave]
{4.24.m}





\end{document}